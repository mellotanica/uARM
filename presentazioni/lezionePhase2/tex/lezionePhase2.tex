\documentclass{beamer}

\usepackage{graphicx}
\usepackage{bytefield}
\usepackage{rotating}

\title{$\mu$ARM, Interrupt e Device}
\author{Marco Melletti \\ melletti.marco@gmail.com}


\newcommand{\colorbitbox}[3]{%
\rlap{\bitbox{#2}{\color{#1}\rule{\width}{\height}}}%
\bitbox{#2}{#3}}

\newcommand{\ruota}[1]{\begin{turn}{90}\footnotesize #1 \end{turn}}

\begin{document}

\frame{\titlepage}

\begin{frame}{Device}
	\begin{itemize}\itemsep10px
		\item Dischi 
		\item Nastri / USB
		\item Interfacce di rete
		\item Stampanti
		\item Terminali
	\end{itemize}

\vfill

\begin{tabular}[t]{rcl}
	\Huge 8 x [ & \Huge \includegraphics[width=0.08\textwidth]{disk-32.png}, 
	\includegraphics[width=0.08\textwidth]{tape-32.png},
	\includegraphics[width=0.08\textwidth]{network-32.png},
	\includegraphics[width=0.08\textwidth]{printer-32.png},
	\includegraphics[width=0.08\textwidth]{terminal-32.png} & \Huge ] \\
\end{tabular}
\end{frame}

\begin{frame}{Dove sono i Device}

base address: (\texttt{0x40 + \emph{dev\_type} * 256 + \emph{dev\_num} * 32}) ...
\vspace{5px}

arch.h: \texttt{DEV\_REG\_ADDR(line, dev)} !

\vfill

Device Register Generali:
	
{	\small
	\begin{tabular}{|c|c|c|}
		\hline
		Fileld n. & Address & Field Name \\
		\hline
		\hline
		0 & (base) + 0x0 & \textbf{STATUS}\\
		\hline
		1 & (base) + 0x4 & \textbf{COMMAND} \\
		\hline
		2 & (base) + 0x8 & \textbf{DATA0} \\
		\hline
		3 & (base) + 0xC & \textbf{DATA1} \\
		\hline
	\end{tabular}
}

\vfill

Device Register Terminali:

{	\small
	\begin{tabular}{|c|c|c|}
			\hline
			Fileld n. & Address & Field Name \\
			\hline
			\hline
			0 & (base) + 0x0 & \textbf{RECV\_STATUS}\\
			\hline
			1 & (base) + 0x4 & \textbf{RECV\_COMMAND} \\
			\hline
			2 & (base) + 0x8 & \textbf{TRANSM\_STATUS} \\
			\hline
			3 & (base) + 0xC & \textbf{TRANSM\_COMMAND} \\
			\hline
		\end{tabular}
}
\end{frame}

\begin{frame}{Interrupts}
Iterrupt vector:

\vspace{5px}
{
	\small
	\begin{tabular}{r|l}
		\texttt{0x6FE0} & Disks \\
		\texttt{0x6FE4} & Tapes \\
		\texttt{0x6FE8} & Network \\
		\texttt{0x6FEC} & Printers \\
		\texttt{0x6FF0} & Terminals \\
	\end{tabular}
}

\vfill
e.g. Terminals:

\vspace{5px}
\begin{bytefield}{32}
\begin{leftwordgroup}{\footnotesize\texttt{0x0000.6FF0}}\bitheader[endianness=big]{31,8,7,6,5,4,3,2,1,0} \\
\colorbitbox{gray}{24}{unused}
\bitbox{1}{\ruota{t7}}
\bitbox{1}{\ruota{t6}}
\bitbox{1}{\ruota{t5}}
\bitbox{1}{\ruota{t4}}
\bitbox{1}{\ruota{t3}}
\bitbox{1}{\ruota{t2}}
\bitbox{1}{\ruota{t1}}
\bitbox{1}{\ruota{t0}}
\end{leftwordgroup}
\end{bytefield}

\vspace{10px}

se {\small (t5 == 1) $=>$} interrupt pendente sulla linea del terminale 5

\end{frame}

\begin{frame}{Altre aree interessanti}
\small
\begin{tabular}{r|l}
	Address & Function \\
	\hline
	\texttt{0x00000020} & Installed Devices Vector \\
	\\
	\texttt{0x000002DC} & Time of Day (Hi) \\
	\texttt{0x000002E0} & Time of Day (Low) \\
	\texttt{0x000002E4} & Interval Timer \\
	\\
	\texttt{0x00007000} & Exception States Vector \\
\end{tabular}

\vfill

Exception States Vector: \hfill(uARMconst.h)

\vspace{5px}
\begin{tabular}{r|c|c|}
\cline{2-3}
\footnotesize\texttt{0x7000} & INTERRUPT\_OLD & INTERRUPT\_NEW \\
\cline{2-3}
\footnotesize\texttt{0x70B0} & TLB\_OLD & TLB\_NEW \\
\cline{2-3}
\footnotesize\texttt{0x7160} & PGMTRAP\_OLD & PGMTRAP\_NEW \\
\cline{2-3}
\footnotesize\texttt{0x7210} & SYSBP\_OLD & SYSBP\_NEW \\
\cline{2-3}
\end{tabular}
\end{frame}

\begin{frame}{Lavorare con Interrupt e Device}
Proviamo a modificare la funzione \texttt{tprint()} in versione "polling" in modo da utilizzare gli interrupt... 
\end{frame}

\begin{frame}{Riferimenti}
Riferimento principale:

\begin{itemize}
\item $\mu$ARM Informal Specifications \small(\url{http://mellotanica.github.io/uARM/uarmdoc.pdf})
\end{itemize}

\vfill
Per i dettagli sui device:

\begin{itemize}
\item $\mu$MPS Principles of Operation \small(\url{http://www.cs.unibo.it/~renzo/so/princOfOperations.pdf})
\end{itemize}

\vfill
Funzioni e strutture di supporto:

\begin{itemize}
\item \texttt{arch.h}, \texttt{uARMconst.h}, \texttt{uARMtypes.h}, \texttt{libuarm.h}
\end{itemize}

\end{frame}

\begin{frame}

\vfill

\center\huge Domande?

\vfill
\normalsize Contattatemi tranquillamente via mailing list (SO) o per email: \url{melletti.marco@gmail.com}

\end{frame}

\end{document}