\usepackage[table]{xcolor}
\usepackage{longtable}
\usepackage{array}
\usepackage{multirow}
\usepackage{ulem}
\usepackage{bytefield}
\usepackage{framed}
\usepackage{adjustbox}
\usepackage{hhline}
\usepackage{listings}
\usepackage{tabularx}

\colorlet{punct}{red!60!black}
\definecolor{background}{gray}{0.95}
\definecolor{delim}{RGB}{20,105,176}
\colorlet{numb}{magenta!60!black}

\lstset{%
 backgroundcolor=\color{background},
 basicstyle=\footnotesize\ttfamily,
 commentstyle=\color{green},
 frame=lines,
 keepspaces=true
}

\lstdefinelanguage{json}{
    literate=
     *{0}{{{\color{numb}0}}}{1}
      {1}{{{\color{numb}1}}}{1}
      {2}{{{\color{numb}2}}}{1}
      {3}{{{\color{numb}3}}}{1}
      {4}{{{\color{numb}4}}}{1}
      {5}{{{\color{numb}5}}}{1}
      {6}{{{\color{numb}6}}}{1}
      {7}{{{\color{numb}7}}}{1}
      {8}{{{\color{numb}8}}}{1}
      {9}{{{\color{numb}9}}}{1}
      {:}{{{\color{punct}{:}}}}{1}
      {,}{{{\color{punct}{,}}}}{1}
      {\{}{{{\color{delim}{\{}}}}{1}
      {\}}{{{\color{delim}{\}}}}}{1}
      {[}{{{\color{delim}{[}}}}{1}
      {]}{{{\color{delim}{]}}}}{1},
}

% restore default italic emph behavior
\makeatletter
\DeclareRobustCommand{\myem}{%
	\@nomath\em
	\ifdim\fontdimen\@ne\font > \z@
	\eminnershape
	\else
	\itshape
\fi}
\DeclareTextFontCommand{\emph}{\myem}
\makeatother

\newcommand{\uarm}{$\mu$ARM}
\newcommand{\register}[1]{\textbf{#1}}
\newcommand{\field}[1]{\textit{#1}}
\newcommand{\addr}[1]{\texttt{#1}}
\newcommand{\asm}[1]{\texttt{\expandafter\MakeUppercase\expandafter{#1}}}
\newcommand{\instr}[1]{\texttt{#1}}

\newcommand{\centerinput}[1]{\vspace{5px}
\begin{adjustbox}{center}
	\input{#1}
\end{adjustbox}
\vspace{5px}}
\newcommand{\spinput}[1]{\centerinput{\specsd/tex/tables/#1}}

%bytefield macros
\newcommand{\colorbitbox}[3]{%
	\rlap{\bitbox{#2}{\color{#1}\rule{\width}{\height}}}%
	\bitbox{#2}{#3}}
% facilitates the creation of memory maps. Start address at the bottom,
% end address at the top.
% syntax:
% \memsection{end address}{start address}{height in lines}{bits of textbox}{text in box}
\newcommand{\memsection}[5]{%
% define the height of the memsection
	\bytefieldsetup{bitheight=#3\baselineskip}%
	\bitbox{#4}{#5}% print box with caption
	\bitbox[]{10}{%
		\addr{#1}% print end address
		\\
		% do some spacing
		\vspace{#3\baselineskip}
		\vspace{-2\baselineskip}
		\vspace{-#3pt}
		\addr{#2}% print start address
	}%
}
\newcommand{\vertbox}[1]{\rotatebox{90}{\footnotesize #1}}
